
\documentclass[12pt,twoside]{article}
\usepackage{amsmath}
\usepackage{amssymb}
\usepackage{listings}
\usepackage{color}
\usepackage{graphicx}
\usepackage{parskip}
\usepackage{hyperref}

\pagestyle{myheadings}
\textwidth 160mm
\textheight 220mm
\oddsidemargin -.2cm
\evensidemargin -.2cm
\markboth{{\rm G.Drummond, R.Cox}}{{\rm {COSC364 Assignment 1}}}


\begin{document}
\title{COSC364 Assignment 2}
\author{George Drummond(53243258), \\Ryan Cox(64656394)}
\maketitle
\thispagestyle{empty}

\begin{abstract}
The following discusses -------
This was a joint work by George Drummond and Ryan Cox in accordance with the course requirements of COSC364-17S1 and is a result of \bf{equal} contribution ($50\% / 50\%$).
\end{abstract}

\tableofcontents

\newpage
\section{ Problem formulation}
We wish to formulate an optimization problem for generic values of X, Y and Z (with $ Y> 3$) such that the load on all transit nodes is balanced.

\subsection{decision and auxiliary variables}
Let $y_{ik}$ be the amount of flow on the link between a given source node $S_i$ and transit node $T_k$.
Likewise let $z_{kj}$ be the amount of flow on the link between a given transit node $T_k$ and destination node $D_j$.

Therefore, letting $x_{ikj}$ be the part of the demand volume between source node $S_i$ and destination node $D_j$ that is routed through transit node $T_k$, we achieve 
\begin{equation}\label{1}
	y_{ik} + z_{kj} = x_{ikj} 
\end{equation}
$\forall i \in [X], k \in [Y], j \in [Z]$.

\subsection{objective function}\label{Sec: objf}
The resulting \emph{utilisation} on a link between a given source node $S_i$ and transit node $T_k$ is given by $\frac{y_{ik}}{c_{ik}}$. Similarly, the resulting utilisation on a link between a given transit node $T_k$ and destination node $D_j$is given by $\frac{z_{kj}}{d_{kj}}$.

We therefore formulate our objective function as.
\begin{equation}\label{2}
	r = \max(\frac{y_{ik}}{c_{ik}},\frac{z_{kj}}{d_{kj}}), \forall i \in [X], k \in [Y], j \in [Z]
\end{equation}

That is, we wish to minimise the greatest link utilisation. We now note that $r$ is piecewise linear in section \ref{Sec: Const} when considering constraints.

\subsection{Demand constraints}
We now turn our attention to the global requirement that each demand volume shall be split over exactly three different paths, such that each path gets an equal
share of the demand volume.

Firstly, we obviously must have the following for the demand volume between a source node $S_i$ and destination node $D_j$.
\begin{equation}\label{3}
	\sum_{k=1}^{Y}x_{ikj} = h_{ij} = i + j
\end{equation}

This however, is somewhat wasteful when we consider the constraint that $x_{ikj}$ must be positive for exactly three values
$k_1, k_2, k_3$  and must be zero for all other $k \in [Y]$.
For each path $P_{ij}$, $i \in [X], j \in [Z]$, we will define,
\begin{align*}
	k_1 &= (j-1) \mod(Y)\\
	k_2 &= j \mod(Y)\\
	k_3 &= (j+1) \mod(Y)
\end{align*}
That is, we have selected those paths through the transit nodes directly 'above' and to the 'above to the side' of the destination node. This ensures an equal coverage EXPLAIN MORE HERE. This achieves the following constraints,
\begin{align*}
	x_{ik_1j} &> 0\\
	x_{ik_2j} &> 0\\
	x_{ik_3j} &> 0
\end{align*}
We would now like to write that $x_{ik_1j} + x_{ik_2j}  + x_{ik_3j} = h_{ij} = i + j$ but in fact, we can do better as these paths must receive an \emph{equal} share of the demand volume, hence,
\begin{align*}
	x_{ik_1j} = x_{ik_2j} = x_{ik_3j}= \frac{h_{ij}}{3} =\frac{i+j}{3} 
\end{align*}
Finally, we have the "no other paths" condition, 
\begin{center}
	$x_{ikj} = 0
	\forall i \in [X], k \in [Y] -\{k_1,k_2,k_3\}, j \in [Z]$
\end{center}


\subsection{Additional constraints}\label{Sec: Const} 
\b{Capacity}:\\
$\forall i \in [X], k \in [Y], j \in [Z]$, we have,
\begin{align*}
	y_{ik} &\leq c_{ik} \\
	z_{kj} &\leq d_{kj}  
\end{align*}

Likewise, from out discussion in \ref{Sec: objf}, we have the following constraints regarding our objective function,
\begin{align*}
	y_{ik} &\leq c_{ik} r\\
	z_{kj} &\leq d_{kj} r
\end{align*}

Non-negativity:\\
$\forall i \in [X], k \in [Y], j \in [Z]$, we have,
\begin{align*}
	x_{ikj} &\geq 0\\
	y_{ik} &\geq 0\\
	z_{kj} &\geq 0\\
	r &\geq 0
\end{align*}

\section{Results}










\definecolor{codegreen}{rgb}{0,0.6,0}
\definecolor{codegray}{rgb}{0.5,0.5,0.5}
\definecolor{codepurple}{rgb}{0.58,0,0.82}
\definecolor{backcolour}{rgb}{0.95,0.95,0.92}

\lstdefinestyle{mystyle}{
	backgroundcolor=\color{backcolour},   
	commentstyle=\color{codegreen},
	keywordstyle=\color{magenta},
	numberstyle=\tiny\color{codegray},
	stringstyle=\color{codepurple},
	basicstyle=\footnotesize,
	breakatwhitespace=false,         
	breaklines=true,                 
	captionpos=b,                    
	keepspaces=true,                 
	numbers=left,                    
	numbersep=5pt,                  
	showspaces=false,                
	showstringspaces=false,
	showtabs=false,                  
	%tabsize=2
}

\lstset{style=mystyle}

\newpage
\section{LP generation source file}
\lstinputlisting[language=python]{genLP.py}
\newpage


\section{References}\label{Sec: Ref}

\end{document}



%%%%%%%%%%%%%%%%%%%%%%%%%%%%%%%%%%%%%%%%%% Non-relevent stuff below this line %%%%%%%%%%%%%%%%%%%%%%%%%%%%%%%%%%%%%